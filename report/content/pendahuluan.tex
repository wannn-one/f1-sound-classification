\section{Pendahuluan}
\subsection{Latar belakang}

Kemajuan teknologi otomotif telah mengarah pada pengembangan mesin berperforma tinggi, khususnya di balap Formula 1 (F1). Suara khas yang dihasilkan oleh mesin F1 memainkan peran penting baik dalam analisis performa maupun pengalaman penonton. Kemampuan untuk mengklasifikasikan dan menganalisis suara mesin secara akurat dapat memberikan wawasan berharga untuk diagnostik mesin, pengoptimalan performa, dan bahkan meningkatkan pengalaman balap secara keseluruhan. \newline

Metode klasifikasi suara tradisional sering mengandalkan inspeksi manual dan penilaian subjektif, yang dapat memakan waktu dan rentan terhadap kesalahan. Dalam beberapa tahun terakhir, penerapan teknik pemrosesan sinyal digital dan algoritma pembelajaran mesin telah menunjukkan hasil yang menjanjikan dalam mengotomatiskan klasifikasi dan analisis suara mesin. \newline

\textit{Fast Fourier Transform} (FFT) adalah teknik pemrosesan sinyal yang banyak digunakan yang memungkinkan kita menganalisis komponen frekuensi dari sinyal suara yang diberikan. Dengan mentransformasi bentuk gelombang suara dari domain waktu ke domain frekuensi, kita dapat mengekstrak fitur penting yang mencirikan suara mesin, seperti komponen frekuensi dominan dan struktur harmonik. \newline

\textit{Deep Learning}, subbidang pembelajaran mesin, telah mendapatkan perhatian yang signifikan karena kemampuannya untuk secara otomatis mempelajari representasi hierarkis dari data yang kompleks. \textit{Convolutional Neural Networks} (CNNs), sejenis model pembelajaran mendalam, telah menunjukkan performa luar biasa dalam berbagai tugas klasifikasi audio. Dengan memanfaatkan CNN, kami dapat membangun sistem klasifikasi suara mesin yang kuat dan akurat yang dapat membedakan antara kondisi dan kondisi mesin yang berbeda berdasarkan fitur frekuensi yang diekstraksi. \newline

Dalam penelitian ini, kami bertujuan untuk mengembangkan sistem klasifikasi suara untuk suara silinder mesin F1 menggunakan kombinasi FFT dan teknik deep learning. Dengan menganalisis konten frekuensi suara mesin dan melatih model CNN pada kumpulan data besar sampel suara mesin berlabel, kami berupaya mencapai akurasi klasifikasi tinggi dan performa waktu nyata. Sistem yang diusulkan memiliki potensi untuk membantu insinyur balap, mekanik, dan penggemar balap dalam mendiagnosis masalah mesin, memantau kinerja, dan meningkatkan pengalaman balap F1 secara keseluruhan.

\subsection{Rumusan masalah}

Berdasarkan latar belakang di atas, maka kami merumuskan masalah sebagai berikut:

\begin{enumerate}
    \item Bagaimana peran FFT dalam pemrosesan sinyal suara?
    \item Bagaimana algoritma \textit{deep learning} yang optimal untuk melakukan klasifikasi suara?
    \item Bagaimana hasil dari berbagai metode \textit{deep learning} mempengaruhi output dari program?
\end{enumerate}

\subsection{Tujuan Penelitian}

Tujuan dari dilakukannya penelitian ini adalah sebagai berikut:

\begin{enumerate}
    \item Mengetahui peran FFT dalam pemrosesan sinyal suara.
    \item Mengetahui algoritma \textit{deep learning} yang optimal dalam melakukan klasifikasi suara
    \item Mengetahui hasil dari berbagai metode \textit{deep learning} dan pengaruhnya terhadap output program
\end{enumerate}

\subsection{Manfaat}

Manfaat yang bisa diberikan oleh penelitian ini adalah bisa menjadi sumber pembelajaran dan bahan bacaan untuk penelitian penelitian lain tentang pemrosesan sinyal dan \textit{Deep Learning} 

\subsection{Metode Penelitian}

Metode yang digunakan untuk penelitian ini yaitu metode kuantitatif, dengan mengambil dataset suara dari \textit{YouTube} berupa suara-suara \textit{onboard camera} dari berbagai musim di F1. Yang kemudian dataset tersebut di-\textit{program} menggunakan bahasa pemrograman \textit{python}

\newpage